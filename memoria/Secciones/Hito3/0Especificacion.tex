\section{Especificación Sintáctica (Gramática)}

Implementamos la gramática que define la especificación sintáctica del lenguaje Tiny empleando los patrones explicados en clase 
(Diseño descendente, Reutilización, Nivel de Abstracción Equilibrado, Opcionalidad, Variantes, Listas y Expresiones). 

Para ello definimos primero la estructura básica de todo programa:

\begin{math}
    programa \longrightarrow \; bloque \\
	programa.a \; = \; mkprogram(bloque.a) \\
    bloque \longrightarrow \; \{ seccion\_declaraciones\_opt \;\; seccion\_instrucciones\_opt \} \\
	bloque.a \; = \; mkbloq(seccion\_declaraciones\_opt.a, seccion\_instrucciones\_opt.a)
\end{math}

\subsection{Declaraciones}

\begin{math}
    seccion\_declaraciones\_opt \longrightarrow \; seccion\_declaraciones \; \&\& \\
        seccion\_declaraciones\_opt.a \; = \; mkdecs(seccion_declaraciones.v) \\
    seccion\_declaraciones\_opt \longrightarrow \; \epsilon \\
        seccion\_declaraciones\_opt.a \; = \; \epsilon \; \\
    seccion\_declaraciones \longrightarrow \; seccion\_declaraciones \; ; \; declaracion \\
        seccion\_declaraciones.a \; = \; mkmult\_decs(seccion\_declaraciones.a, declaracion.a) \\
    seccion\_declaraciones \longrightarrow \; declaracion \\
        seccion\_declaraciones.a \; = \; declaracion.a \\
    declaracion \longrightarrow \; tipo\_nombre \\
        declaracion.a \; = \; tipo\_nombre.a \\
    declaracion \longrightarrow \; \textbf{type} \; tipo\_nombre \\
        declaracion.a \; = \; mktype\_decl(tipo\_nombre.a) \\
    declaracion \longrightarrow \; \textbf{proc} \; \textbf{identificador} \; parametros\_formales \; bloque\\
        declaracion.a \; = \; mkproc(parametros\_formales.a, bloque.a) \\
    parametros\_formales \longrightarrow \; ( lista\_parametros\_opt ) \\
        parametros\_formales.a \; = \; lista\_parametros\_opt.a
    lista\_parametros\_opt \longrightarrow \; lista\_parametros \\
        lista\_parametros\_opt.a \; = \; lista\_parametros.a \\
    lista\_parametros\_opt \longrightarrow \; \epsilon \\
        lista\_parametros\_opt.a \; = \; \epsilon \\
    lista\_parametros \longrightarrow \; lista\_parametros \; , \; parametro \\
        lista\_parametros.a \; = \; mklparam(lista\_parametros.a, parametro.a)
    lista\_parametros \longrightarrow \; parametro \\
        lista\_parametros.a \; = \; parametro.a \\
    parametro \longrightarrow \; tipo \; ref\_opt \; \textbf{identificador} \\
        parametro.a \; = \; mksparam(tipo.a, ref\_opt.a, valorDe(\textbf{identificador}))
    ref\_opt \longrightarrow \; \& \\
        ref\_opt.a = '&'
    ref\_opt \longrightarrow \; \epsilon \\
        ref\_opt.a = ''
\end{math}

\subsection{Tipos}

\begin{math}
    tipo\_nombre \longrightarrow \; tipo \; \textbf{identificador} \\
        tipo\_nombre.a \; = \; mktnombre(tipo.a, valorDe(\textbf{identificador})) \\
    tipo \longrightarrow \; tipo0 \\
        tipo.a \; = \; tipo0.a \\
    tipo0 \longrightarrow \; tipo0 \; [\textbf{literalEntero}] \\
        tipo0.a \; = \; mkarray(tipo0.a, \textbf{literalEntero}) \\
    tipo0 \longrightarrow \; tipo1 \\
        tipo0.a \; = \; tipo1.a \\
    tipo1 \longrightarrow \; \hat{} \; tipo1 \\
        tipo1.a \; = \; mkpuntero(tipo1.a)
    tipo1 \longrightarrow \; tipo\_base \\
        tipo1.a \; = \; tipo\_base.a \\
    tipo\_base \longrightarrow \; \textbf{struct} \; \{ lista\_campos \} \\
        tipo\_base.a \; = \; mkstruct(lista\_campos.a) \\
    tipo\_base \longrightarrow \; \textbf{int} \\
        tipo\_base.a \; = \; valorDe(\textbf{int}) \\
    tipo\_base \longrightarrow \; \textbf{real} \\
        tipo\_base \; = \; valorDe(\textbf{real}) \\
    tipo\_base \longrightarrow \; \textbf{bool} \\
        tipo\_base \; = \; valorDe(\textbf{bool}) \\
    tipo\_base \longrightarrow \; \textbf{string} \\
        tipo\_base \; = \; valorDe(\textbf{string}) \\
    tipo\_base \longrightarrow \; \textbf{identificador} \\
        tipo\_base \; = \; valorDe(\textbf{identificador}) \\
    lista\_campos \longrightarrow \; lista\_campos \; , \; tipo\_nombre \\
        lista\_campos.a \; = \; mklcampos(lista\_campos, tipo\_nombre.a) \\
    lista\_campos \longrightarrow \; tipo\_nombre \\
        tipo\_nombre.a \; = \; tipo\_nombre.a
\end{math}

\subsection{Instrucciones}

\begin{math}
    seccion\_instrucciones\_opt \longrightarrow \; seccion\_instrucciones \\
        seccion\_instrucciones\_opt.a \; = \; seccion\_instrucciones.a \\
    seccion\_instrucciones\_opt \longrightarrow \; \epsilon \\
        seccion\_instrucciones\_opt.a \; = \; \epsilon \\
    seccion\_instrucciones \longrightarrow \; lista\_instrucciones \\
        seccion\_instrucciones.a \; = \; lista\_instrucciones.a \\
    lista\_instrucciones \longrightarrow \; lista\_instrucciones \; ; \; instruccion \\
        lista\_instrucciones.a \; = \; mklinstr(lista\_instrucciones.a, instruccion.a) \\
    lista\_instrucciones \longrightarrow \; instruccion \\
        lista\_instrucciones.a \; = \; instruccion.a \\
    instruccion \longrightarrow \; @ \; expresion \\
        instruccion.a \; = \; expresion.a \\
    instruccion \longrightarrow \; if\_ins \\
        instruccion.a \; = \; if\_ins.a \\
    instruccion \longrightarrow \; if\_ins \; else\_ins \\
        instruccion.a \; = \; mkifelse(if\_ins.a, else\_ins.a) \\
    instruccion \longrightarrow \; \textbf{while} \; exp\_bloque \\
        instruccion.a \; = \; mkwhile(exp\_bloque.a) \\
    instruccion \longrightarrow \; \textbf{read} \; expresion \\
        instruccion.a \; = \; mkread(expresion.a) \\
    instruccion \longrightarrow \; \textbf{write} \; expresion \\
        instruccion.a \; = \; mkwrite(expresion.a) \\
    instruccion \longrightarrow \; \textbf{nl} \\
        instruccion.a \; = \; mknl() \\
    instruccion \longrightarrow \; \textbf{new} \; expresion \\
        instruccion.a \; = \; mknew(expresion.a) \\
    instruccion  \longrightarrow \; \textbf{delete} \; expresion \\
        instruccion .a \; = \; mkdelete(expresion.a) \\
    instruccion \longrightarrow \; \textbf{call} \; \textbf{identificador} \; parametros\_reales \\
        instruccion.a \; = \; mkcall(valorDe(\textbf{identificador}), parametros\_reales.a) \\
    instruccion \longrightarrow \; bloque \\
        instruccion.a \; = \; bloque.a \\
    if\_ins \longrightarrow \; \textbf{if} \; exp\_bloq.a \\
        if\_ins.a \; = \; mkif(exp\_bloq.a) \\
    else\_ins \longrightarrow \; \textbf{else} \; bloque \\
        else\_ins.a \; = \; bloque.a \\
    exp\_bloq \longrightarrow \; expresion \; bloque \\
        exp\_bloq.a \; = \; mkexpbloq(expresion.a, bloque.a) \\
    parametros\_reales \longrightarrow \; ( lista\_expresiones\_opt ) \\
        parametros\_reales.a \; = \; lista\_expresiones\_opt.a \\
    lista\_expresiones\_opt \longrightarrow \; lista\_expresiones \\
        lista\_expresiones\_opt.a \; = \; lista\_expresiones.a \\
    lista\_expresiones\_opt \longrightarrow \; \epsilon \\
        lista\_expresiones\_opt.a \; = \; \epsilon \\
    lista\_expresiones \longrightarrow \; lista\_expresiones \; , \; expresion \\
        lista\_expresiones.a \; = \; mklexpr(lista\_expresiones.a, expresion.a) \\
    lista\_expresiones \longrightarrow \; expresion
        lista\_expresiones.a \; = \; expresion.a
\end{math}

\subsection{Expresiones}

\begin{math}
    expresion \longrightarrow \; E0 \\
        expresion.a \; = \; E0.a \\
    E0 \longrightarrow \; E1 \; = \; E0 \\
	E0.a \; = \; mkassign(E1.a, E0.a) \\
    E0 \longrightarrow \; E1 \\
        E0.a \; = \; E1.a \\
    E1 \longrightarrow \; E1 \; op\_relacional \; E2 \\
	E1.a \; = \; mkrel(E1.a, E2.a) \\
    E1 \longrightarrow \; E2 \\
        E1.a \; = \; E2.a \\
    E2 \longrightarrow \; E2 \; + \; E3 \\
	E2.a \; = \; mkadd(E2.a, E3.a) \\
    E2 \longrightarrow \; E3 \; - \; E3 \\
	E2.a \; = \; mksub(E3.a, E3.a) \\
    E2 \longrightarrow \; E3 \\
        E2.a \; = \; E3.a \\
    E3 \longrightarrow \; E4 \; \textbf{and} \; E3 \\
	E3.a \; = \; mkand(E4, E3) \\
    E3 \longrightarrow \; E4 \; \textbf{or} \; E4 \\
	E3.a \; = \; mkor(E4.a, E4.a) \\
    E3 \longrightarrow \; E4 \\
        E3.a \; = \; E4.a \\
    E4 \longrightarrow \; E4 \; op\_mult \; E5 \\
	E4.a \; = \; mkmul(E4.a, E5.a) \\
    E4 \longrightarrow \; E5 \\
        E4.a \; = \; E5.a \\
    E5  \longrightarrow \; - \; E5 \\
	E5.a  \; = \; mkneg(E5.a) \\
    E5 \longrightarrow \; \textbf{not} \; E5 \\
	E5.a \; = \; mknot(E5.a) \\
    E5 \longrightarrow \; E6 \\
        E5.a \; = \; E6.a \\
    E6 \longrightarrow \; E6 \; op\_dirs \\
        E6.a \; = \; E6 \; op\_dirs \\
    E6 \longrightarrow \; E7 \\
        E6.a \; = \; E7.a \\
    E7 \longrightarrow \; expresion\_basica \\
        E7.a \; = \; expresion\_basica.a \\
    E7 \longrightarrow \; (E0) \\
        E7.a \; = \; E0.a \\
    expresion\_basica \longrightarrow \; \textbf{literalEntero} \\
	expresion\_basica.a \; = \; valorDe(\textbf{literalEntero}) \\
    expresion\_basica \longrightarrow \; \textbf{literalReal} \\
	expresion\_basica.a \; = \; valorDe(\textbf{literalReal}) \\
    expresion\_basica \longrightarrow \; \textbf{true} \\
	expresion\_basica.a \; = \; valorDe(\textbf{true}) \\
    expresion\_basica \longrightarrow \; \textbf{false} \\
	expresion\_basica.a \; = \; valorDe(\textbf{false}) \\
    expresion\_basica \longrightarrow \; \textbf{literalCadena} \\
	expresion\_basica.a \; = \; valorDe(\textbf{literalCadena}) \\
    expresion\_basica \longrightarrow \; \textbf{identificador} \\
	expresion\_basica.a \; = \; valorDe(\textbf{identificador}) \\
    expresion\_basica \longrightarrow \; \textbf{null} \\
	expresion\_basica.a \; = \; valorDe(\textbf{null})
\end{math}

\subsection{Operadores}

\begin{math}
    op\_relacional \longrightarrow \; < \\
        op\_relacional.a \; = \; < \\
    op\_relacional \longrightarrow \; <= \\
        op\_relacional.a \; = \; <= \\
    op\_relacional \longrightarrow \; > \\
        op\_relacional.a \; = \; > \\
    op\_relacional \longrightarrow \; >= \\
        op\_relacional.a \; = \; >= \\
    op\_relacional \longrightarrow \; == \\
        op\_relacional.a \; = \; == \\
    op\_relacional \longrightarrow \; != \\
        op\_relacional.a \; = \; != \\
    op\_mult \longrightarrow \; * \\
        op\_mult.a \; = \; '*' \\
    op\_mult \longrightarrow \; / \\
        op\_mult.a \; = \; '/' \\
    op\_mult \longrightarrow \% \\
        op\_mult.a \; = \; '\%' \\
    op\_dirs \longrightarrow \; [ expresion ] \\
        op\_dirs.a \; = \; mkindex(expresion.a) \\
    op\_dirs \longrightarrow \; . \; \textbf{identificador} \\
        op\_dirs.a \; = \; mkmember(\textbf{identificador}) \\
    op\_dirs \longrightarrow \; \hat{} \\
        op\_dirs.a \; = \; mkderef(\hat{})
\end{math}
