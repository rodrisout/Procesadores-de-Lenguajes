\section{Especificación Sintáctica (Gramática)}

Implementamos la gramática que define la especificación sintáctica del lenguaje Tiny empleando los patrones explicados en clase 
(Diseño descendente, Reutilización, Nivel de Abstracción Equilibrado, Opcionalidad, Variantes, Listas y Expresiones). 

Para ello definimos primero la estructura básica de todo programa:

\begin{math}
    programa \longrightarrow \; bloque \\
    bloque \longrightarrow \; \{ seccion\_declaraciones\_opt \;\; seccion\_intrucciones\_opt \}
\end{math}

\subsection{Declaraciones}

\begin{math}
    seccion\_declaraciones\_opt \longrightarrow \; seccion\_declaraciones \; \&\& \\
    seccion\_declaraciones\_opt \longrightarrow \; \epsilon \\
    seccion\_declaraciones \longrightarrow \; seccion\_declaraciones \; ; \; declaracion \\
    seccion\_declaraciones \longrightarrow \; declaracion \\
    declaracion \longrightarrow \; tipo\_nombre \\
    declaracion \longrightarrow \; \textbf{type} \; tipo\_nombre \\
    declaracion \longrightarrow \; \textbf{proc} \; \textbf{identificador} \; parametros\_formales \; bloque\\
    parametros\_formales \longrightarrow \; ( lista\_parametros\_opt ) \\
    lista\_parametros\_opt \longrightarrow \; lista\_parametros \\
    lista\_parametros\_opt \longrightarrow \; \epsilon \\
    lista\_parametros \longrightarrow \; lista\_parametros \; , \; parametro \\
    lista\_parametros \longrightarrow \; parametro \\
    parametro \longrightarrow \; tipo \; ref\_opt \; \textbf{identificador} \\
    ref\_opt \longrightarrow \; \& \\
    ref\_opt \longrightarrow \; \epsilon
\end{math}

\subsection{Tipos}

\begin{math}
    tipo\_nombre \longrightarrow \; tipo \; \textbf{identificador} \\
    tipo \longrightarrow \; tipo\_base \\
    tipo \longrightarrow \; tipo \; [\textbf{literalEntero}] \\
    tipo\_base \longrightarrow \; \hat{} \; tipo\_base \\
    tipo\_base \longrightarrow \; \textbf{struct} \; { lista\_campos } \\
    tipo\_base \longrightarrow \; \textbf{int} \\
    tipo\_base \longrightarrow \; \textbf{real} \\
    tipo\_base \longrightarrow \; \textbf{bool} \\
    tipo\_base \longrightarrow \; \textbf{string} \\
    tipo\_base \longrightarrow \; \textbf{identificador} \\
    lista\_campos \longrightarrow \; lista\_campos \; , \; tipo\_nombre \\
    lista\_campos \longrightarrow \; tipo\_nombre
\end{math}

\subsection{Instrucciones}

\begin{math}
    seccion\_intrucciones\_opt \longrightarrow \; seccion\_intrucciones \\
    seccion\_intrucciones\_opt \longrightarrow \; \epsilon \\
    seccion\_intrucciones \longrightarrow \; lista\_instrucciones \\
    lista\_instrucciones \longrightarrow \; lista\_instrucciones \; ; \; instruccion \\
    lista\_instrucciones \longrightarrow \; instruccion \\
    instruccion \longrightarrow \; @ \; expresion \\
    instruccion \longrightarrow \; if\_ins \\
    instruccion \longrightarrow \; if\_ins \; else\_ins \\
    instruccion \longrightarrow \; \textbf{while} \; exp\_bloque \\
    instruccion \longrightarrow \; \textbf{read} \; expresion \\
    instruccion \longrightarrow \; \textbf{write} \; expresion \\
    instruccion \longrightarrow \; \textbf{nl} \\
    instruccion \longrightarrow \; \textbf{new} \; expresion \\
    instruccion  \longrightarrow \; \textbf{delete} \; expresion \\
    instruccion \longrightarrow \; \textbf{call} \; \textbf{identificador} \; parametros_reales \\
    instruccion \longrightarrow \; bloque \\
    if\_ins \longrightarrow \; \textbf{if} \; exp\_bloq \\
    else\_ins \longrightarrow \; \textbf{else} \; bloque \\
    exp\_bloq \longrightarrow \; expresion \; bloque \\
    parametros\_reales \longrightarrow \; ( lista\_expresiones\_opt ) \\
    lista\_expresiones\_opt \longrightarrow \; lista\_expresiones \\
    lista\_expresiones\_opt \longrightarrow \; \epsilon \\
    lista\_expresiones \longrightarrow \; lista\_expresiones \; , \; expresion \\
    lista\_expresiones \longrightarrow \; expresion
\end{math}

\subsection{Expresiones}

\begin{math}
    expresion \longrightarrow \; E0 \\
    E0 \longrightarrow \; E1 \; = \; E0 \\
    E0 \longrightarrow \; E1 \\
    E1 \longrightarrow \; E1 \; op\_relacional \; E2 \\
    E1 \longrightarrow \; E2 \\
    E2 \longrightarrow \; E2 \; + \; E3 \\
    E2 \longrightarrow \; E3 \; - \; E3 \\
    E2 \longrightarrow \; E3 \\
    E3 \longrightarrow \; E4 \; \textbf{and} \; E3 \\
    E3 \longrightarrow \; E4 \; \textbf{or} \; E4 \\
    E3 \longrightarrow \; E4 \\
    E4 \longrightarrow \; E4 \; op\_mult \; E5 \\
    E4 \longrightarrow \; E5 \\
    E5  \longrightarrow \; - \; E5 \\
    E5 \longrightarrow \; \textbf{not} \; E5 \\
    E5 \longrightarrow \; E6 \\
    E6 \longrightarrow \; E6 \; op\_dirs \\
    E6 \longrightarrow \; E7 \\
    E7 \longrightarrow \; expresion\_basica \\
    E7 \longrightarrow \; (E0) \\
    expresion\_basica \longrightarrow \; \textbf{literalEntero} \\
    expresion\_basica \longrightarrow \; \textbf{literalReal} \\
    expresion\_basica \longrightarrow \; \textbf{true} \\
    expresion\_basica \longrightarrow \; \textbf{false} \\
    expresion\_basica \longrightarrow \; \textbf{literalCadena} \\
    expresion\_basica \longrightarrow \; \textbf{identificador} \\
    expresion\_basica \longrightarrow \; \textbf{null}
\end{math}

\subsection{Operadores}

\begin{math}
    op\_relacional \longrightarrow \; < \\
    op\_relacional \longrightarrow \; <= \\
    op\_relacional \longrightarrow \; > \\
    op\_relacional \longrightarrow \; >= \\
    op\_relacional \longrightarrow \; == \\
    op\_relacional \longrightarrow \; != \\
    op\_mult \longrightarrow \; * \\
    op\_mult \longrightarrow \; / \\
    op\_mult \longrightarrow \% \\
    op\_dirs \longrightarrow \; [ expresion ] \\
    op\_dirs \longrightarrow \; . \; \textbf{identificador} \\
    op\_dirs \longrightarrow \; \hat{}
\end{math}