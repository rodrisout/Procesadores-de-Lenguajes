\section{Acondicionamiento}

Acondicionamos la gramática definida en la sección anterior. Ésto, con el fin de implementar un analizador
sintáctico descendente predictivo recursivo.

\begin{math}
    programa \longrightarrow \; bloque \\
    bloque \longrightarrow \; \{ seccion\_declaraciones\_opt \;\; seccion\_instrucciones\_opt \}
\end{math}

\subsection{Declaraciones}

\begin{math}
    seccion\_declaraciones\_opt \longrightarrow \; seccion\_declaraciones \; \&\& \\
    seccion\_declaraciones\_opt \longrightarrow \; \epsilon \\
    % RECURSION A IZQ 
    seccion\_declaraciones \longrightarrow \; declaracion \; resto\_sd \\
    resto\_sd \longrightarrow \; ; \; declaracion \; resto\_sd \\
    resto\_sd \longrightarrow \; \epsilon \\
    %
    declaracion \longrightarrow \; tipo\_nombre
\end{math}

\subsection{Tipos}

\begin{math}
    tipo\_nombre \longrightarrow \; tipo\_base \; \textbf{identificador} \\
    tipo\_base \longrightarrow \; \textbf{int} \\
    tipo\_base \longrightarrow \; \textbf{real} \\
    tipo\_base \longrightarrow \; \textbf{bool}
\end{math}

\subsection{Instrucciones}

\begin{math}
    seccion\_instrucciones\_opt \longrightarrow \; seccion\_instrucciones \\
    seccion\_instrucciones\_opt \longrightarrow \; \epsilon \\
    seccion\_instrucciones \longrightarrow \; lista\_instrucciones \\
    % RECURSION A IZQ
    lista\_instrucciones \longrightarrow \; instruccion \; resto\_li \\
    resto\_li \longrightarrow \; ; \; instruccion \; resto\_li \\
    resto\_li \longrightarrow \; \epsilon \\
    %
    instruccion \longrightarrow \; @ \; expresion
\end{math}

\subsection{Expresiones}

\begin{math}
    expresion \longrightarrow \; E0 \\
    % FACTOR COMUN
    E0 \longrightarrow \; E1 \; resto\_E0 \\
    resto\_E0 \longrightarrow \; = \; E0 \\
    resto\_E0 \longrightarrow \; \epsilon \\
    %
    % RECURSION A IZQ
    E1 \longrightarrow \; E2 \; resto\_E1 \\
    resto\_E1 \longrightarrow \; op\_relacional \; E2 \; resto\_E1 \\
    resto\_E1 \longrightarrow \; \epsilon \\
    %
    % FACTOR COMUN y RECURSION A IZQ
    E2 \longrightarrow \; E3 \; resto\_E2\_F resto\_E2\_R \\
    resto\_E2\_R \longrightarrow \; + \; E3 \; resto\_E2\_R \\
    resto\_E2\_R \longrightarrow \; \epsilon \\
    resto\_E2\_F \longrightarrow \; - \; E3 \\
    resto\_E2\_F \longrightarrow \; \epsilon \\
    %
    % FACTOR COMUN
    E3 \longrightarrow \; E4 \; resto\_E3 \\
    resto\_E3 \longrightarrow \; \textbf{and} \; E3 \\
    resto\_E3 \longrightarrow \; \textbf{or} \; E4 \\
    resto\_E3 \longrightarrow \; \epsilon \\
    %
    % RECURSION A IZQ
    E4 \longrightarrow \; E5 \; resto\_E4 \\
    resto\_E4 \longrightarrow \; op\_mult \; E5 \; resto\_E4 \\
    resto\_E4 \longrightarrow \epsilon \\
    %
    E5  \longrightarrow \; - \; E5 \\
    E5 \longrightarrow \; \textbf{not} \; E5 \\
    E5 \longrightarrow \; E6 \\
    E6 \longrightarrow \; expresion\_basica \\
    E6 \longrightarrow \; (E0) \\
    expresion\_basica \longrightarrow \; \textbf{literalEntero} \\
    expresion\_basica \longrightarrow \; \textbf{literalReal} \\
    expresion\_basica \longrightarrow \; \textbf{true} \\
    expresion\_basica \longrightarrow \; \textbf{false} \\
    expresion\_basica \longrightarrow \; \textbf{identificador}
\end{math}

\subsection{Operadores}

\begin{math}
    op\_relacional \longrightarrow \; < \\
    op\_relacional \longrightarrow \; <= \\
    op\_relacional \longrightarrow \; > \\
    op\_relacional \longrightarrow \; >= \\
    op\_relacional \longrightarrow \; == \\
    op\_relacional \longrightarrow \; != \\
    op\_mult \longrightarrow \; * \\
    op\_mult \longrightarrow \; /
\end{math}