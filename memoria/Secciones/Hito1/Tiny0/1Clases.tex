\section{Clases léxicas}

\subsection{Palabras reservadas}

\t Para poder analizar de manera correcta, será necesario establecer una clase léxica por cada palabra reservada. En el lenguaje de
esta práctica, \textit{Tiny (0)}, contamos con 6 palabras reservadas, 3 de ellas utilizadas para definir el tipo de las variables.
Tendremos pues, una palabra para las variables de tipo booleano, otra para las de tipo entero y una última para las reales. Además
de éstas tendremos 3 palabras utilizadas para las operaciones lógicas. Las palabras son las definidas a continuación, contando cada con
una clase léxica.

\begin{itemize}
    \item \textit{bool} $\rightarrow$ Variables booleanas.
    \item \textit{int} $\rightarrow$ Variables enteras.
    \item \textit{real} $\rightarrow$ Variables reales.
    \item \textit{and} $\rightarrow$ Conjunción lógica.
    \item \textit{or} $\rightarrow$ Disyunción lógica.
    \item \textit{not} $\rightarrow$ Negación lógica.
    \item \textit{true} $\rightarrow$ Valor booleano cierto.
    \item \textit{false} $\rightarrow$ Valor booleano falso.
\end{itemize}

\subsection{Literales}

\begin{itemize}
    \item \textbf{Literales enteros.} Opcionalmente empiezan con un signo más (+) o menos (-), y después debe aparecer una
        secuencia (que empieza por un número distinto de 0) de 1 o más dígitos. Su clase léxica será \textit{literalEntero}.
    \item \textbf{Literales reales.}Empieza con una parte entera seguida bien de una parte decimal, bien de una exponecial o bien una parte decimal seguida de exponecial. La parte decimal comienza con el signo punto (.) seguido de una secuencia (que puede ser sólo un 0 o números que no acaben en 0) de 1 o más dígitos. La parte exponencial se indica con (e) o (E), seguida de una parte entera. Su clase léxica será \textit{literalReal}.
\end{itemize}

\subsection{Identificadores}

Los identificadores nos sirven para poder ponerle un nombre a las variables. Éstos deben comenzar por un subrayado (\_) o una letra, seguida de una secuencia de 0 o más
subrayados, dígitos o letras. Su clase léxica será \textit{identificador}.

\subsection{Símbolos de operación y puntuación}

Cada uno de ellos tendrá su propia clase léxica. En el subconjunto del lenguaje en el que trabajamos, \textit{Tiny (0)}, contamos con
las siguientes clases:

\begin{itemize}
    \item \textbf{Suma.} Se representa con el símbolo más (+). Su clase léxica será \textit{suma}.
    \item \textbf{Resta.} Se representa con el símbolo símbolo menos (-). Su clase léxica será \textit{resta}.
    \item \textbf{Multiplicación.} Se representa con el símbolo asterisco (*). Su clase léxica será \textit{mul}.
    \item \textbf{División.} Se representa con el símbolo barra (/). Su clase léxica será \textit{div}.
    \item \textbf{Menor.} Se representa con el símbolo menor que (<). Su clase léxica será \textit{menor}.
    \item \textbf{Mayor.} Se representa con el símbolo mayor que (>). Su clase léxica será \textit{mayor}.
    \item \textbf{Igual.} Se representa con dos símbolos de igualdad seguidos (==). Su clase léxica será \textit{igual}.
    \item \textbf{Menor o igual.} Se representa con el símbolo menor que seguido del símbolo de igualdad (<=). Su clase léxica será \textit{menorIgual}.
    \item \textbf{Mayor o igual.} Se representa con el símbolo mayor que seguido del símbolo de igualdad (>=). Su clase léxica será \textit{mayorIgual}.
    \item \textbf{Asignación.} Se representa con un símbolo de igualdad (=). Su clase léxica será \textit{asig}.
    \item \textbf{Final.} Se representa con el símbolo ampersand dos veces consecutivas (\&\&). Su clase léxica será \textit{finalAsig}.
    \item \textbf{Paréntesis de apertura.} Se representa con el símbolo del paréntesis de apertura (``('', sin comillas). Su clase léxica será \textit{parenApert}.
    \item \textbf{Paréntesis de cierre.} Se representa con el símbolo del paréntesis de cierre (``)'', sin comillas). Su clase léxica será \textit{parenCierre}.
    \item \textbf{Llave de apertura.} Se representa con el símbolo de la llave de apertura (``\{'', sin comillas). Su clase léxica será \textit{LlaveApert}.
    \item \textbf{Llave de cierre.} Se representa con el símbolo de la llave de cierre (``\}'', sin comillas). Su clase léxica será \textit{LlaveCierre}.
    \item \textbf{Punto y coma.} Se representa con el símbolo punto y coma (;). Su clase léxica será \textit{puntoComa}.
    \item \textbf{Arroba.} Se representa con el símbolo arroba (@). Su clase léxica será \textit{arroba}.
\end{itemize}
